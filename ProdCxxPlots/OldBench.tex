%%%%%%%%%%%%%%%%%%%%%%%%%%%%%%%%%%%%%%%%%%%%%%%%%%%%%%%%%%%%%%%%
%%%%%%%%%%%%%%%%%%%%%%%%%%%%%%%%%%%%%%%%%%%%%%%%%%%%%%%%%%%%%%%%
%%%%%%%%%%%%%%%%%%%%%%%%%%%%%%%%%%%%%%%%%%%%%%%%%%%%%%%%%%%%%%%%
\documentclass[12pt]{article}


%%%%%%%%%%%%%%%%%%%%%%%%%%%%%%%%%%%%%%%%%%%%%%%%%%%%%%%%%%%%%%%%%
\usepackage{epsfig}

\usepackage{amsmath}
\usepackage{amssymb}
\usepackage{euscript}

\usepackage{fancybox}
\usepackage{xcolor}


%%%%%%%%%%%%%%%%%%%%%%%%%%%%%%%%%%%%%%%%%%%%%%%%%%%%
%\input Energy.tex
%\input Process.tex
\def\Energy{92.3GeV}
\def\Process{$e^+e^-\to e^+e^-$}


%%%%%%%%%%%%%%%%%%%%%%%%%%%%%%%%%%%%%%%%%%%%%%%%%%%%%%%%%%%%%%%
%  copied from seminar.con
\def\titbox#1{\begin{center}\doublebox{#1}\end{center}}

\newcommand{\KK}{${\cal KK}$}
%--------------------------------------------------------------
\def\Order#1{${\cal O}(#1)$}
\def\Ordpr#1{${\cal O}(#1)_{prag}$}
\def\Oceex#1{${\cal O}(#1)_{_{\rm CEEX}}$}
\def\Oeex#1{${\cal O}(#1)_{_{\rm EEX}}$}
\def\OrderLL#1{${\cal O}(#1)_{\rm LL}$}


%%%%%%%%%%%%%%%%%%%%%%%%%%%%%%%%%%%%%%%%%%%%%%%%%%%%%%%%%%%%%%%
\textwidth  = 16cm % <-- maximum CERN
\textheight = 22cm % <-- maximum CERN
\hoffset    = -1cm
\voffset    = -1cm


%%%%%%%%%%%%%%%%%%%%%%%%%%%%%%%%%%%%%%%%%%%%%%%%%%%%%%%
%%%%%%%%%%%%%%%%%%%%%%%%%%%%%%%%%%%%%%%%%%%%%%%%%%%%%%%
\begin{document}                     %%%%%%%%%%%%%%%%%%


%//////////////////////////////////////////////////////////////////////////////////
%//////////////////////////////////////////////////////////////////////////////////
%//////////////////////////////////////////////////////////////////////////////////

\section{Baseline benchmark, \today}
WARNING: Tables are from machine produced LaTeX source.

\begin{table}[!h]
\begin{center}
{\epsfig{file=TabOldBench.pdf,width=130mm}}
\end{center}
\caption{\sf
{\bf Results from BHLUMI Monte Carlo for \Process\ at \Energy.}
Reproducing Tables 14 and 16 in Proc. of workshop 1996.
%%%%Here VP and Z are switched OFF.
%
Here VP and Z are switched ON.
%%%%VP according to Burkhardt and Pietrzyk 1995 (Moriond).
S. Eidelman, F. Jegerlehner, Z. Phys. C (1995)
%
Energy cut on $z_{\min}=s'/s$ for BARE1
and $z_{\min}=4 E_1 E_2 /s$ for CALO2 and SICAL2.
%
%%%%Statistics 5G events (2h run on 24 processors).
%%%%Statistics 8G events (3h run on 24 processors).
Statistics 3G events (1h run on 24 processors).
}
\end{table}

%%%%%%%%%%%%%%%%%%%%%%%%%%%%%%%%%%%%%%%%%%%%%%%%%%%%%%%%%%%%%%%%%%%%%%%%%%%%%%%%%%%%
%%%%%%%%%%%%%%%%%%%%%%%%%%%%%%%%%%%%%%%%%%%%%%%%%%%%%%%%%%%%%%%%%%%%%%%%%%%%%%%%%%%%
%%%%%%%%%%%%%%%%%%%%%%%%%%%%%%%%%%%%%%%%%%%%%%%%%%%%%%%%%%%%%%%%%%%%%%%%%%%%%%%%%%%%
\newpage
\section{Vacuum polarization study...}

\begin{table}[!h]
\begin{center}
{\epsfig{file=TabVP1.pdf,width=160mm}}
\end{center}
\caption{\sf
Four types of total (hadronic + leptonic) VP function $\Pi(t)$
calculated using routines of the BHLUMI MC.
Angular range is that of narrow-narrow selection of the SICAL detector,
i.e. with $\theta \in (0.02825, 0.0495)$.
Here we use $\sqrt{s}=93.2$GeV for historical reasons.
}
\end{table}

\begin{table}[!h]
\begin{center}
{\epsfig{file=TabVP2.pdf,width=80mm}}
\end{center}
\caption{\sf
Born cross section study with VP off (raw No.1) and four types of VP on
calculated using routines of the BHLUMI MC,
together with errors due to VP provided also by these subprograms.
VPs in the raws No. 2-5 are coming 
in the same order as in previous Table.
(Helmut89, Fred95, Bolek95, Fred17).
The range of $t$ is again that of narrow-narrow SICAL acceptance.
In the 2nd colum VP is averaged over angles while in 3rd column
it is taken at the median transfer $t_{med} = 2/(1/t_1+1/t_2)$.
Here we use $\sqrt{s}=93.2$GeV for historical reasons.
}
\end{table}

%%%%%%%%%%%%%%%%%%%%%%%%%%%%%%%%%%%%%%%%%%%%%%%%%
\newpage
\begin{table}[!h]
\begin{center}
{\epsfig{file=TabVP3.pdf,width=160mm}}
\end{center}
\caption{\sf
Full table of vacuum polarization used in all versions of BHLUMI 
and its error in the entire angular range of all LEP detectors.
This reproduced Tables 5 and 6 in workshop 1996
(except last column which is new).
The error in Burkhardt89 was set to be 4\% of the hadronic part.
Here we use $\sqrt{s}=91.224$GeV to translate from angle to transfer $t$
(as in workshop96).
}
\end{table}

\end{document}



